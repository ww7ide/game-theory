\documentclass[oneside,a4paper]{ctexbook}
\usepackage{geometry}

\usepackage{amsthm}
\usepackage{amsmath,mathtools}
\usepackage{amssymb}
\usepackage{mathrsfs}

\usepackage{enumitem}

\usepackage{hyperref}

\usepackage{tikz}
\usepackage{pgfplots}

\usepackage{float}

\usepackage{multirow}

\theoremstyle{plain}
\newtheorem*{definition}{定义}
\newtheorem*{axiom}{公理}
\newtheorem*{proposition}{命题}
\newtheorem*{theorem}{定理}
\newtheorem*{corollary}{推论}

\theoremstyle{definition}
\newtheorem*{remark}{注}
\newtheorem*{example}{例}

\newcounter{problem}[chapter]
\newenvironment{problem}[1][]{
    \refstepcounter{problem}
    \noindent\textbf{\theproblem.\hspace{1em}#1}
    % \par\bigskip
}{}

\newenvironment{proof*}{\begin{proof}[\textbf{\emph{证明.}}]}{\end{proof}}
\newenvironment{solution}{\begin{proof}[\textbf{\emph{解.}}]}{\end{proof}}

\newlist{enumerate_bf_alph}{enumerate}{1}
\setlist[enumerate_bf_alph]{label=\textbf{\alph*.}}

\newlist{enumerate_bf_arabic}{enumerate}{1}
\setlist[enumerate_bf_arabic]{label=\textbf{\arabic*}}

\SetEnumitemKey{question}{label=\textbf{\alph*.}}

\newcommand{\rmd}{\mathrm d}
\newcommand{\bbE}{\mathbb E}
\newcommand{\bbP}{\mathbb P}
\newcommand{\calP}{\mathcal P}
\newcommand{\bbR}{\mathbb R}

%====================================
% ElegantBook 风格封面,重写 maketitle
%====================================
% 需要的宏包
\usepackage{hyperref}
\usepackage{geometry}
\usepackage{tikz}
\usetikzlibrary{calc}

% 颜色
\definecolor{structurecolor}{RGB}{60,113,183}
\definecolor{main}{RGB}{0,166,82}
\definecolor{second}{RGB}{255,134,24}
\definecolor{third}{RGB}{0,174,247}
\colorlet{coverlinecolor}{second}
\definecolor{darkgray}{gray}{0.30}

% 中文字体
\ifcsname heiti\endcsname
    \newcommand{\cbfseries}{\heiti}
\else
    \newcommand{\cbfseries}{\bfseries}
\fi
\ifcsname kaishu\endcsname
    \newcommand{\citshape}{\kaishu}
    \newcommand{\cnormal}{\kaishu}
\else
    \newcommand{\citshape}{\itshape}
    \newcommand{\cnormal}{\normalfont}
\fi

% 标签文字
\newcommand{\authorname}{\citshape 作者:}
\newcommand{\institutename}{\citshape 组织:}
\newcommand{\datename}{\citshape 时间:}
\newcommand{\versionname}{\citshape 版本:}

\makeatletter

% 对外接口命令
\newcommand{\subtitle}[1]{\gdef\@subtitle{#1}}
\newcommand{\institute}[1]{\gdef\@institute{#1}}
\newcommand{\logo}[1]{\gdef\@logo{#1}}
\newcommand{\cover}[1]{\gdef\@cover{#1}}
\newcommand{\extrainfo}[1]{\gdef\@extrainfo{#1}}
\newcommand{\version}[1]{\gdef\@version{#1}}
\newcommand\bioinfo[2]{\gdef\@bioinfo{{\citshape #1}:#2}}

% 重写 maketitle
\renewcommand{\maketitle}{
    \hypersetup{pageanchor=false}
    % \pagenumbering{Alph}
    \begin{titlepage}
        \newgeometry{margin=0in}
        \parindent=0pt

        % 顶部封面图
        \ifcsname @cover\endcsname
            \includegraphics[width=\linewidth]{\@cover}
        \else
            \@empty
        \fi

        % 彩色横条
        % \setlength{\fboxsep}{0pt}
        % \colorbox{coverlinecolor}{\makebox[\linewidth][c]{\shortstack[c]{\vspace{0.5in}}}}

        % 标题
        \vfill
        \vskip 15ex % \vskip-2ex
        \hspace{2em}
        \parbox{0.8\textwidth}{
            \bfseries\Huge
            \ifcsname @title\endcsname \cnormal\@title \fi
            \par
        }

        % 副标题 + 信息
        \vfill
        % \vspace{-1.0cm}
        \hspace{2.5em}
        \begin{minipage}[c]{0.67\linewidth}
            \color{darkgray}
            {\bfseries\Large
            \ifcsname @subtitle\endcsname \cnormal\@subtitle\\[2ex] \fi}
            \color{gray}
            % \normalsize
            {\renewcommand{\arraystretch}{1.2}
            \begin{tabular}{l}
                % \ifcsname @author\endcsname \authorname \@author\\ \fi
                \ifx\@author\empty\else \authorname\cnormal\@author\\ \fi
                \ifcsname @institute\endcsname \institutename\cnormal\@institute\\ \fi
                % \ifcsname @date\endcsname \@date\\ \fi
                \ifx\@date\empty\else \datename\cnormal\@date\\ \fi
                \ifcsname @version\endcsname \versionname\cnormal\@version\\ \fi
                \ifcsname @bioinfo\endcsname \cnormal\@bioinfo\\ \fi
            \end{tabular}}
        \end{minipage}

        % 右下角 logo
        \begin{minipage}[c]{0.27\linewidth}
            \begin{tikzpicture}[remember picture,overlay]
                \node[opacity=0.8,
                    anchor=south east,
                    outer sep=0pt,
                    inner sep=0pt] at
                    ($(current page.south east)+(-0.8in,1.5in)$)
                    {\ifcsname @logo\endcsname \includegraphics[width=4.2cm]{\@logo} \fi};
            \end{tikzpicture}
        \end{minipage}

        % 底部说明文字
        \vfill
        \begin{center}
            \parbox[t]{0.7\textwidth}{\centering\citshape
                \ifcsname @extrainfo\endcsname \@extrainfo \fi
            }
        \end{center}
        \vfill
    \end{titlepage}
    \restoregeometry
    \thispagestyle{empty}
}

\makeatother

\begin{document}

\href{https://en.wikipedia.org/wiki/Prisoner%27s_dilemma}{囚徒困境}

\begin{definition}[优势策略]
    设有一个包含 $n$ 个参与者的博弈,第 $i$ 个参与者的策略集合为 $S_i$ 其收益函数为 $u_i : S_1 \times S_2 \times \cdots \times S_n \to \mathbb{R}$ 且 $s_i, s_i' \in S_i$。若
    \[
    u_i(s_i', s_{-i}) < u_i(s_i, s_{-i}) \,,\quad \forall s_{-i} \in S_{-i} \,,
    \]
    则称 $s_i'$ 严格劣势于 $s_i$ 或 $s_i$ 严格优势于 $s_i'$。若
    \[
    u_i(s_i', s_{-i}) \leq u_i(s_i, s_{-i}) \,,\quad \forall s_{-i} \in S_{-i} \,,
    \]
    且
    \[
    u_i(s_i', s_{-i}) < u_i(s_i, s_{-i}) \,,\quad \exists s_{-i} \in S_{-i} \,,
    \]
    则称 $s_i'$ 弱劣势于 $s_i$ 或 $s_i$ 弱优势于 $s_i'$。
\end{definition}

\href{https://en.wikipedia.org/wiki/Guess_2/3_of_the_average}{猜均值的 $2/3$}

\href{https://en.wikipedia.org/wiki/Rationalizable_strategy}{迭代剔除劣势策略}

\href{https://en.wikipedia.org/wiki/Median_voter_theorem}{中间选民定理}

\begin{definition}[最佳对策]
    设有一个包含 $n$ 个参与者的博弈,第 $i$ 个参与者的策略集合为 $S_i$ 其收益函数为 $u_i : S_1 \times S_2 \times \cdots \times S_n \to \mathbb{R}$。策略 $s_i' \in S_i$ 称为在对手策略组合 $s_{-i} \in S_{-i}$ 下的最佳对策,当且仅当
    \[
    u_i(s_i', s_{-i}) \geq u_i(s_i, s_{-i}) \,,\quad \forall s_i \in S_i \,.
    \]
\end{definition}

\href{}{合伙人博弈}

\href{https://en.wikipedia.org/wiki/Strategic_complements}{策略互补博弈}

如果每个参与者都选择了自身的策略,并且没有参与者能够仅改变自身的策略而其他参与者保持不变而获益,那么当前的策略组合就构成了纳什均衡。

\begin{definition}[纳什均衡]
    设有一个包含 $n$ 个参与者的博弈,第 $i$ 个参与者的策略集合为 $S_i$ 其收益函数为 $u_i : S_1 \times S_2 \times \cdots \times S_n \to \mathbb{R}$。策略组合
    \[
    (s_1^\ast, s_2^\ast, \dots, s_n^\ast)
    \]
    称为纳什均衡,当且仅当对于任意参与者 $i$ 都有
    \[
    u_i(s_i^\ast, s_{-i}^\ast) \geq u_i(s_i, s_{-i}^\ast) \,,\quad \forall s_i \in S_i \,.
    \]
\end{definition}

\href{}{投资者博弈}

\href{https://en.wikipedia.org/wiki/Coordination_game}{协调博弈}

\href{https://en.wikipedia.org/wiki/Battle_of_the_sexes_(game_theory)}{性别战}

\href{https://en.wikipedia.org/wiki/Cournot_competition}{Cournot 竞争}

\href{https://en.wikipedia.org/wiki/Marginal_cost}{边际成本}

\href{https://en.wikipedia.org/wiki/Demand_curve}{需求曲线}

\href{https://en.wikipedia.org/wiki/Marginal_revenue}{边际收益}

\href{https://en.wikipedia.org/wiki/Monopoly}{垄断}

\href{https://en.wikipedia.org/wiki/Imperfect_competition}{不完全竞争}

\href{https://en.wikipedia.org/wiki/Perfect_competition}{完全竞争}

\href{https://en.wikipedia.org/wiki/Strategic_complements}{策略替代博弈}

\href{https://en.wikipedia.org/wiki/Bertrand_competition}{Bertrand 竞争}

\end{document}