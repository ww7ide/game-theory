\chapter{}

\href{https://en.wikipedia.org/wiki/Bertrand_competition}{Bertrand 竞争}

\section*{Problem Set 3}

\begin{problem}[The Linear City: Price Competition with Differentiated Products]
    在课堂上,我们讨论了两种双寡头竞争模型:Cournot(产量)竞争和 Bertrand(价格)竞争。看起来把企业视为以价格而非产量竞争更为现实,但 Cournot 的结果又比 Bertrand 的结果更 ``符合直觉''。本题考虑第三种双寡头竞争模型。像 Bertrand 那样,两家公司将以价格而不是产量进行竞争。但与 Bertrand 模型不同的是,两家公司的产品并非同质。用经济学术语说,产品是差异化的。我不会在黑板上为你们推导该模型,而是让你们自己来解。不过别慌:习题组会一步步引导你。

    \begin{itemize}
        \item 我们把一座 ``城市'' 看作一条长度为 $1$ 的线段。
        \item 有两家企业,1 和 2,分别位于这条线段的两端。
        \begin{itemize}
            \item 两家企业同时选择价格,分别为 $p_1$ 和 $p_2$。
            \item 两家企业的边际成本都相同,记为 $c$。
            \item 每个企业的目标都是最大化自己的利润。
        \end{itemize}
        \item 潜在消费者均匀分布在线段上,每个点上有一个消费者。
        \begin{itemize}
            \item 设总消费者数量为 $1$(或者如果你愿意,也可以理解为市场份额)。
        \end{itemize}
        \item 每个潜在消费者都只买 $1$ 单位的产品,购买对象只能是企业 1 或企业 2。因此,总需求始终正好为 $1$。
        \item 考虑位于线段上位置为 $y$ 的消费者。他距离企业 1 的距离是 $y$,距离企业 2 的距离是 $1−y$。
        \begin{itemize}
            \item 位于位置 $y$ 的消费者会选择购买企业 1 的产品,如果
            \begin{align}\label{Pset_3_TLC_E1}
            p_1 + t y^2 < p_2 + t (1 - y)^2 \,;
            \end{align}
            会选择购买企业 2 的产品,如果
            \begin{align}\label{Pset_3_TLC_E2}
            p_1 + t y^2 > p_2 + t (1 - y)^2 \,;
            \end{align}
            如果两边刚好相等,那么他将公平地抛硬币决定购买哪一家。
        \end{itemize}
    \end{itemize}

    消费者同时在乎价格和与企业之间的 ``距离''。如果我们把这条线理解为地理距离,那么 $t \times \text{距离}^2$ 这一项就可以理解为消费者去到该企业的 ``交通成本''。或者,如果我们把这条线理解为产品质量的一种维度——比如冰淇淋的脂肪含量——那么这项就是消费者为了购买自己最理想的产品而付出的 ``不便利成本''。当交通成本参数 $t$ 越大时,我们可以认为在消费者眼中两家企业的产品越具有差异化。如果 $t=0$,那么两种产品就是完全替代品。

    \begin{enumerate_bf_alph}
        \item\label{Pset_3_TLC_Q1} 企业 $i$ 会不会设定其价格 $p_i < c$?为什么?
        \item\label{Pset_3_TLC_Q2} 假设企业 2 设置了价格 $p_2$。在什么价格下企业 1 能够拿下整个市场(也就是说,给定 $p_2$,设定什么 $p_1$ 会让所有消费者都从企业 1 购买)?
    \end{enumerate_bf_alph}

    让我们来考虑一下,企业 1 是否可以通过将价格设定得比 \ref{Pset_3_TLC_Q2} 的答案更高而获得更高利润。企业 1 提高价格的坏处在于它会失去一部分市场份额。好处在于对于仍然留下来的顾客,它可以收取更高的价格。接下来的问题将帮助你计算当两家企业的价格 ``很接近'' 时,会有多少顾客选择从企业 1 购买。

    \begin{enumerate_bf_alph}[start=3]
        \item 假设价格 $p_1$ 和 $p_2$ 足够接近,使得市场被两家企业分割(不一定是均分)。利用 \eqref{Pset_3_TLC_E1} 和 \eqref{Pset_3_TLC_E2} 找出那个对从企业 1 购买和从企业 2 购买完全无差异的消费者的位置。利用你的答案来说明,当市场被分割时,企业 1 的需求由下式给出:
        \begin{align}\label{Pset_3_TLC_E3}
            D_1(p_1, p_2) = \frac{p_2 + t - p_1}{2t} \,.
        \end{align}
    \end{enumerate_bf_alph}

    现在我们已经拥有所有需要的信息来计算在每个给定的 $p_2$ 下,企业 1 的最佳对策。当市场被分割时,企业 1 的利润由下式给出:
    \begin{align}\label{Pset_3_TLC_E4}
        u_1(p_1, p_2) = p_1 D_1(p_1, p_2) - c D_1(p_1, p_2) \,,
    \end{align}
    其中第一项是收入,第二项是成本。

    \begin{enumerate_bf_alph}[start=4]
        \item 利用 \eqref{Pset_3_TLC_E3} 和 \eqref{Pset_3_TLC_E4},再结合一些简单的微积分,证明对于中间水平的 $p_2$,有
        \begin{align}
            BR_1(p_2) = \frac{p_2 + t + c}{2} \,.
        \end{align}
        \item\label{Pset_3_TLC_Q5} 画出企业 1 和企业 2 的最佳对策曲线。在图中标明当 $p_2 < c - t$ 和 $p_2 > 3t + c$ 时 $BR_1(p_2)$ 会怎样。(提示:回忆你对 \ref{Pset_3_TLC_Q1} 和 \ref{Pset_3_TLC_Q2} 的回答。)
        \item 使用代数方法求出纳什均衡。
        \item 当 $t = 0$ 时,均衡价格是多少?请解释你的答案。人们有时会说:``随着产品变得不那么相似和更具差异化,竞争会变得不那么激烈。'' 在我们的模型中,这句话是如何体现出来的?
    \end{enumerate_bf_alph}
\end{problem}

\begin{solution}
    \begin{enumerate_bf_alph}
        \item 不会。
        \item 要确定什么 $p_1$ 会让所有消费者都从企业 1 购买,只用让 $y = 1$ 处的消费者从企业 1 购买。令 $y = 1$ 代入 \eqref{Pset_3_TLC_E1} 可知
        \[
        p_1 < p_2 - t
        \]
        时,$y = 1$ 处的消费者会从企业 1 购买。
        \item 由 \eqref{Pset_3_TLC_E1} 和 \eqref{Pset_3_TLC_E2} 可知,当
        \[
        p_1 + t y^2 = p_2 + t (1 - y)^2
        \]
        时,$y$ 处的消费者从企业 1 购买和从企业 2 购买完全无差异,解得其位置
        \[
        y = \frac{p_2 + t - p_1}{2t} \,.
        \]
        同时,任意
        \[
        y < \frac{p_2 + t - p_1}{2t}
        \]
        的消费者都会选择购买企业 1 的产品。因此,可以得出企业 1 的需求
        \[
        D_1(p_1, p_2) = \frac{p_2 + t - p_1}{2t} \,.
        \]
        \item 将 \eqref{Pset_3_TLC_E3} 代入 \eqref{Pset_3_TLC_E4} 整理可得
        \[
        u_1(p_1, p_2) = \frac{p_1 p_2 + t p_1 - p_1^2 - c p_2 - ct + c p_1}{2t} \,.
        \]
        要找到给定 $p_2$ 时,企业 1 的最佳对策,即找到 $u_1$ 的最大值点。令
        \[
        \frac{\partial u_1}{\partial p_1} = \frac{p_2 + t - 2p_1 + c}{2t} = 0 \,,
        \]
        解得
        \[
        p_1 = \frac{p_2 + t + c}{2} \,.
        \]
        此时
        \[
        \frac{\partial^2 u_1}{\partial p_1^2} = -\frac{1}{t} < 0 \,,
        \]
        故给定 $p_2$ 时,$p_1 = (p_2 + t + c)/2$ 是 $u_1$ 的最大值点,即
        \[
        BR_1(p_2) = \frac{p_2 + t + c}{2} \,.
        \]
        \item 当 $p_2 < c - t$ 时,我们假设
        \[
        p_2 = c - t - \varepsilon \,,
        \]
        其中 $\varepsilon$ 为一个足够小的正数,将 $p_2$ 代入 $BR_1$ 可得此时企业 1 的最佳对策
        \[
        p_1^\ast = c - \frac{\varepsilon}{2} \,,
        \]
        由 \ref{Pset_3_TLC_Q2} 可知
        \[
        p_2 < p_1^\ast - t \,,
        \]
        故企业 2 将拿下整个市场。当 $p_2 > 3t + c$ 时,我们假设
        \[
        p_2 = 3t + c + \varepsilon \,,
        \]
        其中 $\varepsilon$ 为一个足够小的正数,将 $p_2$ 代入 $BR_1$ 可得此时企业 1 的最佳对策
        \[
        p_1^\ast = 2t + c + \frac{\varepsilon}{2} \,,
        \]
        由 \ref{Pset_3_TLC_Q2} 可知
        \[
        p_1^\ast < p_2 - t \,,
        \]
        故企业 1 将拿下整个市场。
        \begin{center}
            \begin{tikzpicture}
                % \draw[dashed] (0,0) grid (5,5);
                \draw[->] (0,0) -- (5,0) node[right]{$p_1$};
                \node[left] at (0,0){$O$};
                \draw[->] (0,0) -- (0,5) node[above]{$p_2$};

                % \node[below] at (1,0){$\frac{t+c}{2}$};
                % \node[below] at (2,0){$\frac{2(t+c)}{2}$};
                % \node[below] at (3,0){$\frac{3(t+c)}{2}$};
                % \node[below] at (4,0){$\frac{4(t+c)}{2}$};
                % \node[left] at (0,1){$\frac{t+c}{2}$};
                % \node[left] at (0,2){$\frac{2(t+c)}{2}$};
                % \node[left] at (0,3){$\frac{3(t+c)}{2}$};
                % \node[left] at (0,4){$\frac{4(t+c)}{2}$};

                \draw (1,0) -- (3,4) node[right]{$BR_1$};
                \draw (0,1) -- (4,3) node[right]{$BR_2$};
            \end{tikzpicture}
        \end{center}
        \item 设纳什均衡为 $(p_1^\ast, p_2^\ast)$。由 \ref{Pset_3_TLC_Q5} 可知,$BR_1$ 和 $BR_2$ 的交点即为纳什均衡,令 $p_1^\ast = p_2^\ast = p^\ast$ 代入 $BR_1$ 可解得
        \[
        p^\ast = c + t \,.
        \]
        \item $p_1 = p_2 = c$。当消费者付出的交通成本高于产品价差时,消费者将选择 ``高价'' 产品。
    \end{enumerate_bf_alph}
\end{solution}